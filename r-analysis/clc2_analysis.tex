\documentclass[]{article}
\usepackage{lmodern}
\usepackage{amssymb,amsmath}
\usepackage{ifxetex,ifluatex}
\usepackage{fixltx2e} % provides \textsubscript
\ifnum 0\ifxetex 1\fi\ifluatex 1\fi=0 % if pdftex
  \usepackage[T1]{fontenc}
  \usepackage[utf8]{inputenc}
\else % if luatex or xelatex
  \ifxetex
    \usepackage{mathspec}
  \else
    \usepackage{fontspec}
  \fi
  \defaultfontfeatures{Ligatures=TeX,Scale=MatchLowercase}
\fi
% use upquote if available, for straight quotes in verbatim environments
\IfFileExists{upquote.sty}{\usepackage{upquote}}{}
% use microtype if available
\IfFileExists{microtype.sty}{%
\usepackage{microtype}
\UseMicrotypeSet[protrusion]{basicmath} % disable protrusion for tt fonts
}{}
\usepackage[margin=1in]{geometry}
\usepackage{hyperref}
\hypersetup{unicode=true,
            pdftitle={CLC2 Blocker Analysis},
            pdfauthor={Adam Lu},
            pdfborder={0 0 0},
            breaklinks=true}
\urlstyle{same}  % don't use monospace font for urls
\usepackage{color}
\usepackage{fancyvrb}
\newcommand{\VerbBar}{|}
\newcommand{\VERB}{\Verb[commandchars=\\\{\}]}
\DefineVerbatimEnvironment{Highlighting}{Verbatim}{commandchars=\\\{\}}
% Add ',fontsize=\small' for more characters per line
\usepackage{framed}
\definecolor{shadecolor}{RGB}{248,248,248}
\newenvironment{Shaded}{\begin{snugshade}}{\end{snugshade}}
\newcommand{\AlertTok}[1]{\textcolor[rgb]{0.94,0.16,0.16}{#1}}
\newcommand{\AnnotationTok}[1]{\textcolor[rgb]{0.56,0.35,0.01}{\textbf{\textit{#1}}}}
\newcommand{\AttributeTok}[1]{\textcolor[rgb]{0.77,0.63,0.00}{#1}}
\newcommand{\BaseNTok}[1]{\textcolor[rgb]{0.00,0.00,0.81}{#1}}
\newcommand{\BuiltInTok}[1]{#1}
\newcommand{\CharTok}[1]{\textcolor[rgb]{0.31,0.60,0.02}{#1}}
\newcommand{\CommentTok}[1]{\textcolor[rgb]{0.56,0.35,0.01}{\textit{#1}}}
\newcommand{\CommentVarTok}[1]{\textcolor[rgb]{0.56,0.35,0.01}{\textbf{\textit{#1}}}}
\newcommand{\ConstantTok}[1]{\textcolor[rgb]{0.00,0.00,0.00}{#1}}
\newcommand{\ControlFlowTok}[1]{\textcolor[rgb]{0.13,0.29,0.53}{\textbf{#1}}}
\newcommand{\DataTypeTok}[1]{\textcolor[rgb]{0.13,0.29,0.53}{#1}}
\newcommand{\DecValTok}[1]{\textcolor[rgb]{0.00,0.00,0.81}{#1}}
\newcommand{\DocumentationTok}[1]{\textcolor[rgb]{0.56,0.35,0.01}{\textbf{\textit{#1}}}}
\newcommand{\ErrorTok}[1]{\textcolor[rgb]{0.64,0.00,0.00}{\textbf{#1}}}
\newcommand{\ExtensionTok}[1]{#1}
\newcommand{\FloatTok}[1]{\textcolor[rgb]{0.00,0.00,0.81}{#1}}
\newcommand{\FunctionTok}[1]{\textcolor[rgb]{0.00,0.00,0.00}{#1}}
\newcommand{\ImportTok}[1]{#1}
\newcommand{\InformationTok}[1]{\textcolor[rgb]{0.56,0.35,0.01}{\textbf{\textit{#1}}}}
\newcommand{\KeywordTok}[1]{\textcolor[rgb]{0.13,0.29,0.53}{\textbf{#1}}}
\newcommand{\NormalTok}[1]{#1}
\newcommand{\OperatorTok}[1]{\textcolor[rgb]{0.81,0.36,0.00}{\textbf{#1}}}
\newcommand{\OtherTok}[1]{\textcolor[rgb]{0.56,0.35,0.01}{#1}}
\newcommand{\PreprocessorTok}[1]{\textcolor[rgb]{0.56,0.35,0.01}{\textit{#1}}}
\newcommand{\RegionMarkerTok}[1]{#1}
\newcommand{\SpecialCharTok}[1]{\textcolor[rgb]{0.00,0.00,0.00}{#1}}
\newcommand{\SpecialStringTok}[1]{\textcolor[rgb]{0.31,0.60,0.02}{#1}}
\newcommand{\StringTok}[1]{\textcolor[rgb]{0.31,0.60,0.02}{#1}}
\newcommand{\VariableTok}[1]{\textcolor[rgb]{0.00,0.00,0.00}{#1}}
\newcommand{\VerbatimStringTok}[1]{\textcolor[rgb]{0.31,0.60,0.02}{#1}}
\newcommand{\WarningTok}[1]{\textcolor[rgb]{0.56,0.35,0.01}{\textbf{\textit{#1}}}}
\usepackage{graphicx,grffile}
\makeatletter
\def\maxwidth{\ifdim\Gin@nat@width>\linewidth\linewidth\else\Gin@nat@width\fi}
\def\maxheight{\ifdim\Gin@nat@height>\textheight\textheight\else\Gin@nat@height\fi}
\makeatother
% Scale images if necessary, so that they will not overflow the page
% margins by default, and it is still possible to overwrite the defaults
% using explicit options in \includegraphics[width, height, ...]{}
\setkeys{Gin}{width=\maxwidth,height=\maxheight,keepaspectratio}
\IfFileExists{parskip.sty}{%
\usepackage{parskip}
}{% else
\setlength{\parindent}{0pt}
\setlength{\parskip}{6pt plus 2pt minus 1pt}
}
\setlength{\emergencystretch}{3em}  % prevent overfull lines
\providecommand{\tightlist}{%
  \setlength{\itemsep}{0pt}\setlength{\parskip}{0pt}}
\setcounter{secnumdepth}{0}
% Redefines (sub)paragraphs to behave more like sections
\ifx\paragraph\undefined\else
\let\oldparagraph\paragraph
\renewcommand{\paragraph}[1]{\oldparagraph{#1}\mbox{}}
\fi
\ifx\subparagraph\undefined\else
\let\oldsubparagraph\subparagraph
\renewcommand{\subparagraph}[1]{\oldsubparagraph{#1}\mbox{}}
\fi

%%% Use protect on footnotes to avoid problems with footnotes in titles
\let\rmarkdownfootnote\footnote%
\def\footnote{\protect\rmarkdownfootnote}

%%% Change title format to be more compact
\usepackage{titling}

% Create subtitle command for use in maketitle
\providecommand{\subtitle}[1]{
  \posttitle{
    \begin{center}\large#1\end{center}
    }
}

\setlength{\droptitle}{-2em}

  \title{CLC2 Blocker Analysis}
    \pretitle{\vspace{\droptitle}\centering\huge}
  \posttitle{\par}
    \author{Adam Lu}
    \preauthor{\centering\large\emph}
  \postauthor{\par}
      \predate{\centering\large\emph}
  \postdate{\par}
    \date{6/11/2019}


\begin{document}
\maketitle

\hypertarget{load-libraries}{%
\subsection{Load libraries}\label{load-libraries}}

\begin{Shaded}
\begin{Highlighting}[]
\CommentTok{# tidyverse includes the packages readr, dplyr, ggplot2 ... etc.}
\KeywordTok{library}\NormalTok{(tidyverse)}

\CommentTok{# ggbeeswarm includes geom_beeswarm()}
\KeywordTok{library}\NormalTok{(ggbeeswarm)}

\CommentTok{# broom includes tidy()}
\KeywordTok{library}\NormalTok{(broom)}

\CommentTok{# ggpubr for paired box plots}
\KeywordTok{library}\NormalTok{(ggpubr)}

\CommentTok{# matlab for fileparts}
\KeywordTok{library}\NormalTok{(matlab)}
\end{Highlighting}
\end{Shaded}

\hypertarget{set-parameters}{%
\subsection{Set parameters}\label{set-parameters}}

\begin{Shaded}
\begin{Highlighting}[]
\CommentTok{## Hard-coded parameters}
\NormalTok{drugFilePath <-}\StringTok{ "../data/blinded/drug-clean/drug-clean_oscDurationSec_averaged.csv"}
\NormalTok{controlFilePath <-}\StringTok{ "../data/blinded/control-clean/control-clean_oscDurationSec_averaged.csv"}
\NormalTok{phaseNamePath <-}\StringTok{ "phase_names.csv"}
\NormalTok{measureLabel <-}\StringTok{ "Oscillation Duration (s)"}
\NormalTok{figTitle <-}\StringTok{ "Change in oscillation duration for each slice"}
\end{Highlighting}
\end{Shaded}

\hypertarget{preparation}{%
\subsection{Preparation}\label{preparation}}

\begin{Shaded}
\begin{Highlighting}[]
\CommentTok{## Extract the file base names}
\NormalTok{drugFileParts <-}\StringTok{ }\KeywordTok{fileparts}\NormalTok{(drugFilePath)}
\NormalTok{drugFileBase <-}\StringTok{ }\NormalTok{drugFileParts}\OperatorTok{$}\NormalTok{name}
\NormalTok{controlFileParts <-}\StringTok{ }\KeywordTok{fileparts}\NormalTok{(controlFilePath)}
\NormalTok{controlFileBase <-}\StringTok{ }\NormalTok{controlFileParts}\OperatorTok{$}\NormalTok{name}

\CommentTok{## Extract the measure name}
\NormalTok{drugFileBaseParts <-}\StringTok{ }\KeywordTok{strsplit}\NormalTok{(drugFileBase, }\StringTok{'_'}\NormalTok{)}
\NormalTok{measureName <-}\StringTok{ }\NormalTok{drugFileBaseParts[[}\DecValTok{1}\NormalTok{]][}\DecValTok{2}\NormalTok{]}
\KeywordTok{paste}\NormalTok{(}\StringTok{"Measure to analyze:"}\NormalTok{, measureName)}
\end{Highlighting}
\end{Shaded}

\begin{verbatim}
## [1] "Measure to analyze: oscDurationSec"
\end{verbatim}

\hypertarget{load-data}{%
\subsection{Load data}\label{load-data}}

\hypertarget{raw-data}{%
\paragraph{Raw data}\label{raw-data}}

\begin{Shaded}
\begin{Highlighting}[]
\NormalTok{drugDataRaw <-}\StringTok{ }\KeywordTok{read_csv}\NormalTok{(drugFilePath)}
\NormalTok{drugDataRaw}
\end{Highlighting}
\end{Shaded}

\begin{verbatim}
## # A tibble: 3 x 11
##   phaseNumber oscDurationSec_~ oscDurationSec_~ oscDurationSec_~
##         <dbl>            <dbl>            <dbl>            <dbl>
## 1           1             7.16             2.04             7.37
## 2           2             6.29             1.86             7.13
## 3           3             5.44             1.93             6.36
## # ... with 7 more variables: oscDurationSec_20190604_slice3 <dbl>,
## #   oscDurationSec_20190604_slice4 <dbl>,
## #   oscDurationSec_20190604_slice7 <dbl>,
## #   oscDurationSec_20190606_slice1 <dbl>,
## #   oscDurationSec_20190606_slice2 <dbl>,
## #   oscDurationSec_20190606_slice3 <dbl>,
## #   oscDurationSec_20190607_slice4 <dbl>
\end{verbatim}

\begin{Shaded}
\begin{Highlighting}[]
\NormalTok{controlDataRaw <-}\StringTok{ }\KeywordTok{read_csv}\NormalTok{(controlFilePath)}
\NormalTok{controlDataRaw}
\end{Highlighting}
\end{Shaded}

\begin{verbatim}
## # A tibble: 3 x 8
##   phaseNumber oscDurationSec_~ oscDurationSec_~ oscDurationSec_~
##         <dbl>            <dbl>            <dbl>            <dbl>
## 1           1             1.84             5.11             1.92
## 2           2             1.73             5.71             1.82
## 3           3             1.74             5.05             1.64
## # ... with 4 more variables: oscDurationSec_20190605_slice3 <dbl>,
## #   oscDurationSec_20190605_slice4 <dbl>,
## #   oscDurationSec_20190606_slice5 <dbl>,
## #   oscDurationSec_20190606_slice7 <dbl>
\end{verbatim}

\begin{Shaded}
\begin{Highlighting}[]
\NormalTok{phaseNameTable <-}\StringTok{ }\KeywordTok{read_csv}\NormalTok{(phaseNamePath)}
\NormalTok{phaseNameTable}
\end{Highlighting}
\end{Shaded}

\begin{verbatim}
## # A tibble: 3 x 2
##   phaseNumber phaseName
##         <dbl> <chr>    
## 1           1 Baseline 
## 2           2 Wash-on  
## 3           3 Wash-out
\end{verbatim}

\hypertarget{reorganized-data}{%
\paragraph{Reorganized data}\label{reorganized-data}}

\begin{Shaded}
\begin{Highlighting}[]
\CommentTok{# Put into tidied format and rename variables}
\NormalTok{drugData <-}\StringTok{ }
\StringTok{    }\NormalTok{drugDataRaw }\OperatorTok\StringTok{ }
\StringTok{    }\KeywordTok{gather}\NormalTok{(}\DataTypeTok{key =} \StringTok{"measureSliceName"}\NormalTok{, }\DataTypeTok{value =} \OperatorTok{!!}\NormalTok{measureName, }\DecValTok{-1}\NormalTok{) }\OperatorTok
\StringTok{    }\KeywordTok{separate}\NormalTok{(}\StringTok{"measureSliceName"}\NormalTok{, }\KeywordTok{c}\NormalTok{(}\StringTok{"measureName"}\NormalTok{, }\StringTok{"date"}\NormalTok{, }\StringTok{"sliceStr"}\NormalTok{), }\DataTypeTok{sep =} \StringTok{"_"}\NormalTok{) }\OperatorTok\StringTok{ }
\StringTok{    }\KeywordTok{unite}\NormalTok{(}\StringTok{"sliceName"}\NormalTok{, }\KeywordTok{c}\NormalTok{(}\StringTok{"date"}\NormalTok{, }\StringTok{"sliceStr"}\NormalTok{), }\DataTypeTok{sep =} \StringTok{"_"}\NormalTok{) }\OperatorTok\StringTok{ }
\StringTok{    }\KeywordTok{left_join}\NormalTok{(phaseNameTable, }\DataTypeTok{by =} \StringTok{"phaseNumber"}\NormalTok{) }\OperatorTok\StringTok{ }
\StringTok{    }\KeywordTok{mutate}\NormalTok{(}\StringTok{"phaseName"}\NormalTok{ =}\StringTok{ }\KeywordTok{factor}\NormalTok{(phaseName))}

\NormalTok{drugData}
\end{Highlighting}
\end{Shaded}

\begin{verbatim}
## # A tibble: 30 x 5
##    phaseNumber measureName    sliceName       oscDurationSec phaseName
##          <dbl> <chr>          <chr>                    <dbl> <fct>    
##  1           1 oscDurationSec 20190603_slice7           7.16 Baseline 
##  2           2 oscDurationSec 20190603_slice7           6.29 Wash-on  
##  3           3 oscDurationSec 20190603_slice7           5.44 Wash-out 
##  4           1 oscDurationSec 20190603_slice9           2.04 Baseline 
##  5           2 oscDurationSec 20190603_slice9           1.86 Wash-on  
##  6           3 oscDurationSec 20190603_slice9           1.93 Wash-out 
##  7           1 oscDurationSec 20190604_slice2           7.37 Baseline 
##  8           2 oscDurationSec 20190604_slice2           7.13 Wash-on  
##  9           3 oscDurationSec 20190604_slice2           6.36 Wash-out 
## 10           1 oscDurationSec 20190604_slice3           5.17 Baseline 
## # ... with 20 more rows
\end{verbatim}

\hypertarget{statistics}{%
\subsection{Statistics}\label{statistics}}

\hypertarget{baseline-vs.-drug}{%
\paragraph{Baseline vs.~Drug}\label{baseline-vs.-drug}}

\begin{Shaded}
\begin{Highlighting}[]
\CommentTok{# Construct formula for t test}
\NormalTok{formulaForTestStr <-}\StringTok{ }\KeywordTok{paste}\NormalTok{(measureName, }\StringTok{"~"}\NormalTok{, }\StringTok{"phaseName"}\NormalTok{)}
\NormalTok{formulaForTest <-}\StringTok{ }\KeywordTok{as.formula}\NormalTok{(formulaForTestStr)}
\KeywordTok{paste}\NormalTok{(}\StringTok{"Paired t test formula:"}\NormalTok{, formulaForTestStr)}
\end{Highlighting}
\end{Shaded}

\begin{verbatim}
## [1] "Paired t test formula: oscDurationSec ~ phaseName"
\end{verbatim}

\begin{Shaded}
\begin{Highlighting}[]
\CommentTok{# Extract only phases 1 and 2}
\NormalTok{drugDataPhase12 <-}\StringTok{ }
\StringTok{    }\NormalTok{drugData }\OperatorTok\StringTok{ }\KeywordTok{filter}\NormalTok{(phaseNumber }\OperatorTok{==}\StringTok{ }\DecValTok{1} \OperatorTok{|}\StringTok{ }\NormalTok{phaseNumber }\OperatorTok{==}\StringTok{ }\DecValTok{2}\NormalTok{)}

\CommentTok{# Apply a paired two sample t-test between phases 1 and 2}
\NormalTok{tTestResultPhase12 <-}
\StringTok{    }\KeywordTok{t.test}\NormalTok{(formulaForTest, }\DataTypeTok{data =}\NormalTok{ drugDataPhase12, }\DataTypeTok{paired =} \OtherTok{TRUE}\NormalTok{)}
\NormalTok{tTestResultPhase12}
\end{Highlighting}
\end{Shaded}

\begin{verbatim}
## 
##  Paired t-test
## 
## data:  oscDurationSec by phaseName
## t = 4.4987, df = 9, p-value = 0.001492
## alternative hypothesis: true difference in means is not equal to 0
## 95 percent confidence interval:
##  0.4026978 1.2173022
## sample estimates:
## mean of the differences 
##                    0.81
\end{verbatim}

\hypertarget{drug-vs.-wash-out}{%
\paragraph{Drug vs.~Wash-Out}\label{drug-vs.-wash-out}}

\begin{Shaded}
\begin{Highlighting}[]
\CommentTok{# Extract only phases 2 and 3}
\NormalTok{drugDataPhase23 <-}\StringTok{ }
\StringTok{    }\NormalTok{drugData }\OperatorTok\StringTok{ }\KeywordTok{filter}\NormalTok{(phaseNumber }\OperatorTok{==}\StringTok{ }\DecValTok{2} \OperatorTok{|}\StringTok{ }\NormalTok{phaseNumber }\OperatorTok{==}\StringTok{ }\DecValTok{3}\NormalTok{)}

\CommentTok{# Apply a paired two sample t-test between phases 2 and 3}
\NormalTok{tTestResultPhase23 <-}
\StringTok{    }\KeywordTok{t.test}\NormalTok{(formulaForTest, }\DataTypeTok{data =}\NormalTok{ drugDataPhase23, }\DataTypeTok{paired =} \OtherTok{TRUE}\NormalTok{)}
\NormalTok{tTestResultPhase23}
\end{Highlighting}
\end{Shaded}

\begin{verbatim}
## 
##  Paired t-test
## 
## data:  oscDurationSec by phaseName
## t = 0.11014, df = 9, p-value = 0.9147
## alternative hypothesis: true difference in means is not equal to 0
## 95 percent confidence interval:
##  -0.7229266  0.7969266
## sample estimates:
## mean of the differences 
##                   0.037
\end{verbatim}

\hypertarget{baseline-vs.-wash-out}{%
\paragraph{Baseline vs.~Wash-Out}\label{baseline-vs.-wash-out}}

\begin{Shaded}
\begin{Highlighting}[]
\CommentTok{# Extract only phases 1 and 3}
\NormalTok{drugDataPhase13 <-}\StringTok{ }
\StringTok{    }\NormalTok{drugData }\OperatorTok\StringTok{ }\KeywordTok{filter}\NormalTok{(phaseNumber }\OperatorTok{==}\StringTok{ }\DecValTok{1} \OperatorTok{|}\StringTok{ }\NormalTok{phaseNumber }\OperatorTok{==}\StringTok{ }\DecValTok{3}\NormalTok{)}

\CommentTok{# Apply a paired two sample t-test between phases 2 and 3}
\NormalTok{tTestResultPhase13 <-}
\StringTok{    }\KeywordTok{t.test}\NormalTok{(formulaForTest, }\DataTypeTok{data =}\NormalTok{ drugDataPhase13, }\DataTypeTok{paired =} \OtherTok{TRUE}\NormalTok{)}
\NormalTok{tTestResultPhase13}
\end{Highlighting}
\end{Shaded}

\begin{verbatim}
## 
##  Paired t-test
## 
## data:  oscDurationSec by phaseName
## t = 2.1594, df = 9, p-value = 0.05912
## alternative hypothesis: true difference in means is not equal to 0
## 95 percent confidence interval:
##  -0.04028771  1.73428771
## sample estimates:
## mean of the differences 
##                   0.847
\end{verbatim}

\hypertarget{plots}{%
\subsection{Plots}\label{plots}}

\begin{Shaded}
\begin{Highlighting}[]
\CommentTok{# Pair the data to visualize the change from baseline to drug}
\KeywordTok{ggpaired}\NormalTok{(drugData, }\DataTypeTok{x =} \StringTok{"phaseName"}\NormalTok{, }\DataTypeTok{y =}\NormalTok{ measureName, }\DataTypeTok{id =} \StringTok{"sliceName"}\NormalTok{,}
    \DataTypeTok{color =} \StringTok{"phaseName"}\NormalTok{, }\DataTypeTok{fill =} \StringTok{"white"}\NormalTok{, }\DataTypeTok{palette =} \KeywordTok{c}\NormalTok{(}\StringTok{"blue"}\NormalTok{, }\StringTok{"red"}\NormalTok{, }\StringTok{"purple"}\NormalTok{), }
    \DataTypeTok{width =} \FloatTok{0.5}\NormalTok{, }\DataTypeTok{point.size =} \FloatTok{1.2}\NormalTok{, }\DataTypeTok{line.size =} \FloatTok{0.5}\NormalTok{, }\DataTypeTok{line.color =} \StringTok{"gray"}\NormalTok{, }
    \DataTypeTok{title =}\NormalTok{ figTitle, }\DataTypeTok{xlab =} \OtherTok{FALSE}\NormalTok{, }\DataTypeTok{ylab =}\NormalTok{ measureLabel, }\DataTypeTok{legend.title =} \StringTok{"Phase Name"}\NormalTok{)}
\end{Highlighting}
\end{Shaded}

\includegraphics{clc2_analysis_files/figure-latex/unnamed-chunk-12-1.pdf}

\begin{Shaded}
\begin{Highlighting}[]
\KeywordTok{ggsave}\NormalTok{(}\KeywordTok{paste}\NormalTok{(measureName, }\StringTok{"pairedboxplot.png"}\NormalTok{, }\DataTypeTok{sep =} \StringTok{"_"}\NormalTok{))}
\end{Highlighting}
\end{Shaded}


\end{document}
